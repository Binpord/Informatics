\documentclass[a4paper]{article}

\usepackage{graphicx}
\usepackage{mathtext}
\usepackage[russian]{babel}
\usepackage[T2A]{fontenc}
\usepackage[utf8]{inputenc}
\usepackage{graphicx}
\usepackage{amssymb}
\usepackage[12pt]{extsizes}

\usepackage[left = 1 cm, top = 1.5 cm, right = 1 cm, bottom = 1.5 cm, bindingoffset = 0.5 cm]{geometry}

\begin{document}

	\section{О проекте}

	Проект представляет из себя калькулятор с распознованием рукописного ввода.
	
	\section{Подробнее}

	Результатом работы является GUI-приложение, позволяющее нарисовать в окне выражение и вычислить его значение. Распознование производится методом опорных векторов (SVM). Для этого приложение делит изображение окна на символы, после чего сжимает их до размеров 28 на 28 пикселей, вычисляет по полученному изображению гистограмму ориентированных градиентов (HOG) и отправляет ее на обработку методу распознования. Метод требует обучения. Для этого при старте, приложение проверяет наличие в рабочей директории файла svm.yml, являющегося результатом обучения, и, если его нет, производит обучение (для этого в рабочей директории должна быть папка training\_files).

	\section{Что было сделано}
	\begin{enumerate}
		\item Была разработана GUI оболочка.
		\item Был разработан метод обучения SVM.
	\end{enumerate}
	
	\section{Чего не было сделано}
	\begin{enumerate}
		\item Не было произведено отдельного исследования, направленного на улучшение точности распознования путем варьирования параметров метода SVM или вычисления HOG.
		\item Не было добавлено распознование дробей.
	\end{enumerate}

	
\end{document}

